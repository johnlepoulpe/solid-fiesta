\documentclass[a4paper, 11pt]{article}%

%%%%%%%%%%%%%%%%%%%%%%%%%%%%%%%%%%%%%%%%%%%%%%%%%%%%%%%%%%%%%%%%%%%%%%%%

% Global structure parameters
\usepackage{fullpage}%

\usepackage[francais]{babel}%

\usepackage[utf8]{inputenc}%
\usepackage[T1]{fontenc}%

% Font selection
% (newpx n'a pas été installé avex TeXLive...)
\usepackage{mathpazo}%
\usepackage{courier}%

% Macro packages
\usepackage{url}%
\usepackage{graphicx}%
\usepackage{listings}%

\usepackage[font=scriptsize,labelfont=bf]{caption}

% Parameters for listings
\lstset{%
	basicstyle=\footnotesize\sffamily,%
	columns=fullflexible,%
	frame=lb,%
	frameround=fftf,%
	language=caml,%
}%

% Fine tuning
\setlength{\parskip}{0.5\baselineskip}%

%%%%%%%%%%%%%%%%%%%%%%%%%%%%%%%%%%%%%%%%%%%%%%%%%%%%%%%%%%%%%%%%%%%%%%%%

\begin{document}

\title{Rapport de projet}

\author{Marco Freire, Clément Legrand-Duchesne}

\date{26 septembre 2017}

\maketitle

\begin{abstract}
  Dans ce rapport seront expliqués les codes du pavage de Penrose
  suivant différents types de parcours, et des tours de Hanoi, problème
  à contrainte résolu récursivement, ainsi que leurs extensions; et les
  choix d'implémentation que nous avons été menés à faire.
  \begin{description}
    
  \item[Mots-clés:] Pavage; parcours; récursivité; contrainte.

   
%\item[Classification ACM:] Voir les codes à l'adresse
%  \begin{center}
%    \url{http://www.acm.org/about/class/class/2012}.
%  \end{center}

  \end{description}
\end{abstract}

%%%%%%%%%%%%%%%%%%%%%%%%%%%%%%%%%%%%%%%%%%%%%%%%%%%%%%%%%%%%%%%%%%%%%%%%

\section{Penrose}
        \subsection{Principe: algorithme de base}
                Ce pavage de Penrose s'obtient par divisions
                successives de triangles aux dimensions particulières. 
                Le pavage de Penrose implémenté utilise deux tuiles de
                base, des triangles d'or. Ce sont des triangles
                isocèles dont la longueur des cotés sont 1 et le
                nombre d'or. 
                Les dimensions de leurs
                côtés sont donc des des nombres flottants. Afin
                d'éviter des erreurs d'arrondi susceptibles de se
                cumuler, et d'alléger le code, nous avons choisi de ne
                travailler qu'avec des flottants, avant des les
                arrondir en entiers juste avant le tracé.
                
                Nous avons choisi de representer les triangles par un
                tableau contenant les coordonnées des trois
                sommets. Par commodité, nous avons choisi de suivre la
                convention suivante: le premier sommet est celui duquel sont issus les deux
                cotés de longeurs égales, le second est le suivant en
                tournant dans le sens horaire.

                Inserer image!

                Afin de différentier les triangles obtus des triangles
                aigus, nous avons créé un type \texttt{triangle}:

		\begin{lstlisting}
                  type triangle = Obtuse | Acute;;
		\end{lstlisting}

                Le travail se résume alors à écrire une fonction \texttt{divide},
                prenant en paramètre un triangle (le tableau des
                coordonnées de ses sommets et son type) ainsi que la
                le nombre de subdivisions restant à effectuer, qui
                effectue la découpe de celui-ci, et s'appellant
                récursivement avec en paramètre les nouveaux triangles
                obtenus.

                L'idée est alors de découper récursivement un triangle
                initial de grande taille. 

                Inserer image!

                Une des principales difficultés rencontrées lors de
                l'implémentation de ce pavage est le fait la fonction
                \texttt{fill\_poly} du module \texttt{Graphics.cma}
                remplit le polygone avec une couleur donnée, mais
                recouvre et donc efface les bords de celui ci si ceux
                ci ont été tracés avant. Cela n'est pas encore trop
                contraignant ici, puisqu'il suffit d'écrire la
                fonction \texttt{draw} de la sorte:

                \begin{lstlisting}
                  let draw points triangle =
                      (if triangle = Obtuse then set_color red
                      else set_color blue);
                      fill_poly (iof_array points);
                      move points.(0);
                      set_color black;
                      line points.(1);  
                      line points.(2);
                      line points.(0)
                  ;;
                \end{lstlisting}
                
	\subsection{Améliorations}

		Le premier problème du code précédent est que les
                côtés des triangles sont tous tracés deux fois. Pour y
                remédier, il suffit, lors de chaque subdivision d'un
                triangle, de ne tracer que les séparations entre les
                nouveaux triangles obtenus (et de ne pas oublier les
                cotés du triangle initial).

		Inserer une image! (+ peuve?)

		Il apparaît alors qu'il est necéssaire de tracer ces
                lignes après que les triangles soient remplis, afin
                que celles ci ne soient pas effacées.
		La fonction \texttt{divide} est en réalité un simple
                parcours en profondeur de l'arbre des divisions des
                triangles.
		Plutôt que d'afficher directement le pavage, il semble
                plus interressant de montrer la division succéssive
                des triangles pendant la construction du
                pavage. Survient alors une nouvelle difficulté:
                l'algorithme de parcours en profondeur décrit
                précédemment ne convient plus.

 
		Une première solution est d'implémenter un parcours
                largeur et de marquer une pause avant d'entamer chaque
                nouvelle profondeur de l'arbre. Il faut pour cela
                maintenir une structure de file. Une des conséquences
                de ce choix d'implémentation est que la
                complexité mémoire est un $\Theta(2^{n})$ où $n$ est le
                nombre de générations et non plus un $\Theta(n)$
                correspondant à la profondeur de la ple de récursion.
                Par ailleurs, il est
                alors nettement plus difficle d'afficher les couleurs
                des triangles dessinés à chaque génération et de ne
                tracer chaque coté qu'une seule fois. En effet, à
                chaque nouvelle génération, la coloration des
                triangles efface les cotés de ceux ci si ils ont été
                précédemment tracés. 
                
                Une seconde solution est d'effectuer plusieurs
                parcours profondeur, en augmentant le nombre de
                génération à chaque fois, et en marquant une pause
                entre chaque nouveau parcours. Cela permet de
                conserver une complexité mémoire linéaire en le nombre
                de générations et d'afficher les couleurs des triangles
                à chaque génération. De plus, il est possible de
                partir d'un triangle initial de petite taille et 
                d'effectuer une homothétie à la fin de chaque
                parcours, afin de donner l'impression que le pavage
                s'expand.
                En revanche, du fait que l'on recommence
                systématiquement le parcours depuis le debut, deux
                fois plus de calculs sont effectués.

                La subdivision d'un grand triangle initial, en
                contrepartie d'une certaine simplicité et efficacité,
                nécessite de prévoir le nombre de générations à l'avance.

\section{Hanoi}

	\subsection{Principe: algorithme de base}
		Le problème des tours de Hanoi est essentiellement récursif. Pour
		déplacer n disques du premier poteau jusqu'au troisième
		il suffit de savoir en déplacer n-1 sur le deuxième poteau (étape 1), puis
		de déplacer le n-ième disque sur le troisième poteau (étape 2) et enfin de
		déplacer encore une fois les n-1 disques sur le troisième poteau (étape 3).

		\begin{figure}[!h]
			\minipage{0.24\textwidth}
			  \includegraphics[width=\linewidth]{hanoi_start.png}
			  \caption{État initial}\label{fig:hanoi_start}
			\endminipage\hfill
			\minipage{0.24\textwidth}
			  \includegraphics[width=\linewidth]{hanoi_mid1.png}
			  \caption{Fin étape 1}\label{fig:hanoi_mid1}
			\endminipage\hfill
			\minipage{0.24\textwidth}
			  \includegraphics[width=\linewidth]{hanoi_mid2.png}
			  \caption{Fin étape 2}\label{fig:hanoi_mid2}
			\endminipage\hfill
			\minipage{0.24\textwidth}
			  \includegraphics[width=\linewidth]{hanoi_end.png}
			  \caption{État final}\label{fig:hanoi_end}
			\endminipage\hfill
		\end{figure}

		Nous avons choisi pour représenter la situation un tableau de piles
		\texttt{rods}: chaque pile du tableau représente chacun des poteaux et
		contient les disques qui y sont empilés.

		La seule complication du code est le choix du poteau temporaire utilisé
		pour chaque déplacement de disques. La première implémentation réalisée
		repose sur la fonction suivante \texttt{choose}:

		\begin{lstlisting}
		let choose a b =
		  if a = b then failwith "a and b are equal"
		  else if a = 0 then
			if b = 1 then 2
			else 1
		  else if a = 1 then
			if b = 0 then 2
			else 0
		  else if a = 2 then
			if b = 0 then 1
			else 0
		  else failwith "a or b are not between 0 and 2"
		;;
		\end{lstlisting}

		Cette fonction prend en argument deux éléments distincts de {0, 1, 2} et
		renvoie le troisième et est utilisée pour choisir automatiquement dans
		la fonction \texttt{move} le poteau temporaire à utiliser:

		\begin{lstlisting}
		let rec move rods num_discs orig_rod dest_rod =
		  if num_discs = 1 then
			  move_disc rods orig_rod dest_rod
		  else
			(
			  let temp_rod = choose orig_rod dest_rod in
			  move rods (num_discs - 1) orig_rod temp_rod;
			  
			  move_disc rods orig_rod dest_rod;
			  
			  move rods (num_discs - 1) temp_rod dest_rod;
			)
		;;
		\end{lstlisting}

		Lors de l'implémentation de la fonction résolvant le problème des tours
		de Hanoi à n poteaux, nous nous sommes rendu compte que la fonction
		\texttt{choose} était difficilement généralisable. Nous avons donc choisi
		de passer le poteau intermédiaire en argument à la fonction \texttt{move}.

		Cette modification permet de choisir comme l'on veut le poteau intermédiaire,
		ce qui est nécessaire pour la version généralisée de l'algorithme.
	
	\subsection{Analyse du problème}
		
		Il est possible de calculer le nombre minimal de déplacements à effectuer
		pour résoudre le problème.
		
		Si l'on veut déplacer n disques du premier piquet au dernier, il
		faut nécessairement déplacer le plus grand. Or il n'est possible
		de le déplacer que si les (n-1) disques plus petits ne sont pas
		sur celui-ci. Puisque l'on veut déplacer le plus grand disque sur
		le dernier piquet, les (n-1) disques doivent se trouver empilés
		en ordre décroissant de taille sur le piquet central; dans quel
		cas on peut déplacer le grand disque sur le dernier poteau, et
		déplacer encore les autres disques sur le dernier piquet.
		
		On a ainsi la relation de récurrence suivante, qui se résout facilement:
		
		\begin{description}
			\item[Initialisation] $M(1) = 1$
			\item[Relation de récurrence] $M(n) = 2M(n-1) + 1$
			\item[Relation générale] $M(n) = 2^n - 1$
		\end{description}
		
		Le nombre de déplacements de disques effectués pas l'algorithme
		dans le cas où il n'y a que trois piquets est optimal.
				
		\begin{figure}[!h]
			\minipage{0.49\textwidth}
			  \includegraphics[width=\linewidth]{steps_plot.pdf}
			\endminipage\hfill
			\minipage{0.49\textwidth}
			  \includegraphics[width=\linewidth]{time_plot.pdf}
			\endminipage\hfill
		\end{figure}
			
		Les résultats expérimentaux sont en accord avec
		le calcul théorique: le nombre de mouvements
		est fonction exponentielle du nombre
		initial de disques.
	
	\subsection{Principe: algorithme généralisé}
		Pour ce problème, l'implémentation choisie est la deuxième présentée.
		
		Cette fois il s'agit d'utiliser au mieux les poteaux supplémentaires.
		Nous avons calculé le nombre de disques pouvant être sur les poteaux
		intermédiaires, pour répartir ces disques sur chacun des poteaux intermédiaires
		en utilisant comme poteau temporaire le dernier (étape 1), ensuite il faut
		déplacer le disque le plus grand sur le dernier poteau (étape 2), et finalement
		il suffit de réaliser le parcours inverse de celui effectué dans l'étape 1,
		et de regrouper les disques intermédiaires sur le dernier poteau (étape 3).
		
		\begin{figure}[!h]
			\minipage{0.24\textwidth}
			  \includegraphics[width=\linewidth]{hanoi_gen_start.png}
			  \caption{État initial}\label{fig:hanoi_gen_start}
			\endminipage\hfill
			\minipage{0.24\textwidth}
			  \includegraphics[width=\linewidth]{hanoi_gen_mid1.png}
			  \caption{Fin étape 1}\label{fig:hanoi_gen_mid1}
			\endminipage\hfill
			\minipage{0.24\textwidth}
			  \includegraphics[width=\linewidth]{hanoi_gen_mid2.png}
			  \caption{Fin étape 2}\label{fig:hanoi_gen_mid2}
			\endminipage\hfill
			\minipage{0.24\textwidth}
			  \includegraphics[width=\linewidth]{hanoi_gen_end.png}
			  \caption{État final}\label{fig:hanoi_gen_end}
			\endminipage\hfill
		\end{figure}
		
		Ainsi beaucoup moins de déplacements sont nécessaires pour la résolution
		du problème.
	
	\subsection{Affichage}
		Nous avons créé un affichage grâce au module \texttt{Graphics}
		de OCaml qui permet de visualiser la résolution du problème à n
		disques et p poteaux pour n et p raisonnables.
		
		\begin{figure}[!h]
			\minipage{0.49\textwidth}
			  \includegraphics[width=\linewidth]{hanoi_10_4.png}
			  \caption{Affichage de \texttt{hanoi 10 4}}
			\endminipage\hfill
			\minipage{0.49\textwidth}
			  \includegraphics[width=\linewidth]{hanoi_35_10.png}
			  \caption{Affichage de \texttt{hanoi 35 10}}
			\endminipage\hfill
		\end{figure}
		
		De plus l'affichage s'adapte à la taille de la fenêtre:
		
		\begin{figure}[!h]
			\minipage{0.49\textwidth}
			\begin{center}
			  \includegraphics[width=7cm]{hanoi_wide.png}
			  \caption{1200 x 200}
			\end{center}
			\endminipage\hfill
			\minipage{0.49\textwidth}
			\begin{center}
			  \includegraphics[height=5cm]{hanoi_tall.png}
			  \caption{350 x 750}
			\end{center}
			\endminipage\hfill
		\end{figure}
	
		La taille des disques est donc variable et dépend de plusieurs
		facteurs: au moins il y a de poteaux et au plus la fenêtre est
		large, au plus les disques sont larges; au plus il y a de disques
		et au plus la fenêtre est grande, au plus les disques sont hauts.
	
	\subsection{Améliorations}
		Nous nous sommes rendus compte qu'il était possible d'améliorer
		l'algorithme généralisé, puisqu'on observe que à chaque fois
		qu'un pile de disques est déplacée, seul le premier ou le dernier
		piquet est utilisé comme piquet temporaire, alors que l'on dispose
		de plusieurs piquets libres.
	
\section*{Conclusion}

	\begin{itemize}
		\item Nous avons implémenté le programme dessinant le pavage de
		Penrose, à l'aide d'un parcours en profondeur ou d'un parcours
		en largeur, avec la possibilité d'utiliser une homothétie pour
		agrandir la figure à chaque génération; puis le programme résolvant
		le problème des tours de Hanoi simple et généralisé, avec affichage
		graphique.
		%\item Ce que vous avez choisi de ne pas faire.
		\item Il serait intéressant d'implémenter pour les pavages de Penrose
		une version où l'un des triangles serait fixe au cours des générations
		afin de créer l'illusion de la construction du motif autour d'un
		seul triangle.
		Pour les tours de Hanoi, il serait possible de corriger l'algorithme
		généralisé et utiliser tous les piquets libres à notre disposition
		afin de résoudre le problème en moins de déplacements.
	\end{itemize}

\appendix

\section*{Auto-Évaluation}

	%Votre auto-évaluation selon le schéma SWOT/FFOM~(\url{https://fr.wikipedia.org/wiki/SWOT}).
	\begin{description}
		\item[Forces.] L'avantage principal de notre implémentation du
                  pavage de Penrose est que le type de parcours
                  effectué présente des avantages différents, et ce
                  choix est par conséquent laissé à l'utilisateur. 
                  Par ailleurs la version finale avec homothétie est
                  assez aboutie et pourrait par exemple être utilisée
                  comme économiseur d'écran.
                  En ce qui concerne la partie consacrée au Tours de Hanoi
                  je pense que le point fort est l'affichage très flexible que nous avons
                  implémenté. 
		\item[Faiblesses.] Pour Penrose, les triangles en
                  dehors de l'écran sont tout de même traités, ce qui
                  ralentis considérablement le code quand le nombre de
                  générations est élevé. D'autre part, l'homothétie
                  pourrait se faire de manière plus progressive, pour
                  donner l'impression d'un zoom continu et fluide.
                  Le manque de modularité et de séparation du code
                  s'est vu très pénible au moment d'harmoniser les différentes versions
                  du code à envoyer (algorithme basique, généralisé et avec affichage).
		\item[Opportunités.] Ce projet m'a appris l'importance de la modularité
                  du code, pour les raisons avant mentionnées, et aussi à gérer un projet
                  mené par plusieurs personnes, ce qui peut s'avérer plus difficile que
                  prévu.
		\item[Menaces.] Il est trop souvent facile de remplir le code de
                  littéraux et de constantes, mais quand il faut généraliser le
                  code déjà écrit, cet effort non effectué fait perdre énormément de temps.
	\end{description}
	
\end{document}

%%%%%%%%%%%%%%%%%%%%%%%%%%%%%%%%%%%%%%%%%%%%%%%%%%%%%%%%%%%%%%%%%%%%%%%%
